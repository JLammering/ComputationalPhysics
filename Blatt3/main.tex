\documentclass[
  bibliography=totoc,     % Literatur im Inhaltsverzeichnis
  captions=tableheading,  % Tabellenüberschriften
  titlepage=firstiscover, % Titelseite ist Deckblatt
  DIV=16
]{scrartcl}

% Paket float verbessern
\usepackage{scrhack}

% Seitenränder etc.
%\usepackage[a4paper]{geometry}

% Warnung, falls nochmal kompiliert werden muss
\usepackage[aux]{rerunfilecheck}

% deutsche Spracheinstellungen
\usepackage{polyglossia}
\setmainlanguage{german}

\usepackage{sourceserifpro}
\usepackage{sourcesanspro}
%\usepackage{charter}

% unverzichtbare Mathe-Befehle
\usepackage{amsmath}
% viele Mathe-Symbole
\usepackage{amssymb}
% Erweiterungen für amsmath
\usepackage{mathtools}


% Fonteinstellungen
\usepackage{fontspec}
% Latin Modern Fonts werden automatisch geladen

\usepackage[
  math-style=ISO,    % ┐
  bold-style=ISO,    % │
  sans-style=italic, % │ ISO-Standard folgen
  nabla=upright,     % │
  partial=upright,   % ┘
  warnings-off={           % ┐
    mathtools-colon,       % │ unnötige Warnungen ausschalten
    mathtools-overbracket, % │
  },                       % ┘
]{unicode-math}


% traditionelle Fonts für Mathematik
%\setmainfont{SourceSerifPro}
%\setmathfont{Latin Modern Math}
%\setmathfont{XITS Math}[range={scr, bfscr}]
%\setmathfont{XITS Math}[range={cal, bfcal}, StylisticSet=1]

% Zahlen und Einheiten
\usepackage[
  locale=DE,                 % deutsche Einstellungen
  separate-uncertainty=true, % immer Fehler mit \pm
  per-mode=reciprocal,       % ^-1 für inverse Einheiten
%  output-decimal-marker=.,   % . statt , für Dezimalzahlen
]{siunitx}

% chemische Formeln
\usepackage[
  version=4,
  math-greek=default, % ┐ mit unicode-math zusammenarbeiten
  text-greek=default, % ┘
]{mhchem}

% richtige Anführungszeichen
\usepackage[autostyle]{csquotes}

% schöne Brüche im Text
\usepackage{xfrac}

% Standardplatzierung für Floats einstellen
\usepackage{float}
\floatplacement{figure}{htbp}
\floatplacement{table}{htbp}

% Floats innerhalb einer Section halten
\usepackage[
  section, % Floats innerhalb der Section halten
  below,   % unterhalb der Section aber auf der selben Seite ist ok
]{placeins}

% Captions schöner machen.
\usepackage[
  labelfont=bf,        % Tabelle x: Abbildung y: ist jetzt fett
  font=small,          % Schrift etwas kleiner als Dokument
  width=0.9\textwidth, % maximale Breite einer Caption schmaler
]{caption}
% subfigure, subtable, subref
\usepackage{subcaption}

% Grafiken können eingebunden werden
\usepackage{graphicx}
% größere Variation von Dateinamen möglich
\usepackage{grffile}

% schöne Tabellen
\usepackage{booktabs}

% Verbesserungen am Schriftbild
\usepackage{microtype}

% Literaturverzeichnis
%\usepackage[
%  backend=biber,
%  sorting=none
%]{biblatex}

% Hyperlinks im Dokument
\usepackage[
  unicode,        % Unicode in PDF-Attributen erlauben
  pdfusetitle,    % Titel, Autoren und Datum als PDF-Attribute
  pdfcreator={},  % ┐ PDF-Attribute säubern
  pdfproducer={}, % ┘
]{hyperref}
% erweiterte Bookmarks im PDF
\usepackage{bookmark}

% Trennung von Wörtern mit Strichen
\usepackage[shortcuts]{extdash}

\usepackage[shortlabels]{enumitem}

\usepackage{braket}

\DeclareMathOperator{\Tr}{Tr}
\usepackage[makeroom]{cancel}

%Eigene Befehle
\ExplSyntaxOn
\NewDocumentCommand \fig {mmmO{}}
{
  \begin{figure}
    \centering
    \includegraphics[#4]{#1}
    \caption{#2}
    \label{fig:#3}
  \end{figure}
}
\ExplSyntaxOff


\author{Jasper Karl Lammering\\Henning Ptaszyk\\Timo Gräßer}
\subtitle{Computational Physics}


\usepackage{listings}
\usepackage[dvipsnames]{xcolor}

\title{Blatt 3}
\date{
  Abgabe: 11.05.18
}

\begin{document}
\maketitle
Wir stimmen für die mündliche Prüfung.

Alle Dateien ausführbar mit der Makefile. %Der Output wird dann in die Datei\texttt{blatt2\_ausgabe.txt} geschrieben.
\section*{Aufgabe 1}
\subsection*{a)}
Allgemeine Formel:
\begin{align}
  <O(E)> &= \frac{\sum_\text{alle Zustände} O \exp(-\beta E)}{Z}\\
  \intertext{mit}
  Z &= \sum_i \exp(-E_i \beta).\\
  \intertext{Hier:}
  m &= <s> = \frac{\sum_s s \exp(-E \beta)}{\sum_s \exp(-E \beta)}\\
  \intertext{mit}
  s &= \pm 1, \quad \beta = 1 \quad \text{und}\quad E = -s H\\
  \\
  => m &= \frac{1\exp(+H) - 1\exp(-H)}{\exp(+H) + \exp(-H)} = \frac{2 \sinh(H)}{2 \cosh(H)} = \tanh(H)
\end{align}

\subsection*{b)}
\begin{figure}
  \centering
  \includegraphics[height=8cm]{build/magnetisierung.pdf}
  \caption{Plot der Magnetisierung in Abhängigkeit von H.}
  \label{}
\end{figure}

\section*{Aufgabe 2}

\subsection*{a)}
\begin{figure*}
        \centering
        \begin{subfigure}[b]{0.475\textwidth}
            \centering
            \includegraphics[width=\textwidth]{build/plots/kbTgleicheins_sweep.pdf}
            \caption[Network2]%
            {{\small Momentaufnahme des Systems bei $k_\text{B}T=1$}}
            \label{fig:lcghist1}
        \end{subfigure}
        \hfill
        \begin{subfigure}[b]{0.475\textwidth}
            \centering
            \includegraphics[width=\textwidth]{build/plots/kbTgleichzwei_sweep.pdf}
            \caption[]%
            {{\small Momentaufnahme des Systems bei $k_\text{B}T=2$.}}
            \label{fig:lcghist2}
        \end{subfigure}
        \vskip\baselineskip
        \begin{subfigure}[b]{0.475\textwidth}
            \centering
            \includegraphics[width=\textwidth]{build/plots/kbTgleichdrei_sweep.pdf}
            \caption[]%
            {{\small Momentaufnahme des Systems bei $k_\text{B}T=3$.}}
            \label{fig:lcghist3}
        \end{subfigure}
    \end{figure*}

\subsection*{b)}
%6 Plots: 3 Temperaturen mit 2 Anfangsbedingungen
\begin{figure*}
        \centering
        \begin{subfigure}[b]{0.475\textwidth}
            \centering
            \includegraphics[width=\textwidth]{build/plots/energiekbTgleicheins.pdf}
            \caption[Network2]%
            {{\small Die Energie des Sytems in der Äquilibrierungsphase bei zufällig geordneten Spins im Anfangszustand und $k_\text{B}T=1$.}}
            \label{fig:lcghist1}
        \end{subfigure}
        \hfill
        \begin{subfigure}[b]{0.475\textwidth}
            \centering
            \includegraphics[width=\textwidth]{build/plots/energiekbTgleicheinsgleich.pdf}
            \caption[]%
            {{\small Die Energie des Sytems in der Äquilibrierungsphase bei gleich geordneten Spins im Anfangszustand und $k_\text{B}T=1$.}}
            \label{fig:lcghist2}
        \end{subfigure}
        \vskip\baselineskip
        \begin{subfigure}[b]{0.475\textwidth}
            \centering
            \includegraphics[width=\textwidth]{build/plots/energiekbTgleichzwei.pdf}
            \caption[]%
            {{\small Die Energie des Sytems in der Äquilibrierungsphase bei zufällig geordneten Spins im Anfangszustand und $k_\text{B}T=2$.}}
            \label{fig:lcghist3}
        \end{subfigure}
        \quad
        \begin{subfigure}[b]{0.475\textwidth}
            \centering
            \includegraphics[width=\textwidth]{build/plots/energiekbTgleichzweigleich.pdf}
            \caption[]%
            {{\small Die Energie des Sytems in der Äquilibrierungsphase bei gleich geordneten Spins im Anfangszustand und $k_\text{B}T=2$.}}
            \label{fig:lcghist4}
        \end{subfigure}

    \end{figure*}

    \begin{figure*}
            \centering
            \begin{subfigure}[b]{0.475\textwidth}
                \centering
                \includegraphics[width=\textwidth]{build/plots/energiekbTgleichdrei.pdf}
                \caption[Network2]%
                {{\small Die Energie des Sytems in der Äquilibrierungsphase bei zufällig geordneten Spins im Anfangszustand und $k_\text{B}T=3$.}}
                \label{fig:lcghist1}
            \end{subfigure}
            \hfill
            \begin{subfigure}[b]{0.475\textwidth}
                \centering
                \includegraphics[width=\textwidth]{build/plots/energiekbTgleichdreigleich.pdf}
                \caption[]%
                {{\small Die Energie des Sytems in der Äquilibrierungsphase bei gleich geordneten Spins im Anfangszustand und $k_\text{B}T=3$.}}
                \label{fig:lcghist2}
            \end{subfigure}
        \end{figure*}

\subsection*{c)}
\begin{figure*}
        \centering
        \begin{subfigure}[b]{0.475\textwidth}
            \centering
            \includegraphics[width=\textwidth]{build/plots/energiekbTgleicheins_sweep.pdf}
            \caption[Network2]%
            {{\small Die Energie des Sytems nach der Äquilibrierungsphase bei zufällig geordneten Spins im Anfangszustand und $k_\text{B}T=1$.}}
            \label{fig:lcghist1}
        \end{subfigure}
        \hfill
        \begin{subfigure}[b]{0.475\textwidth}
            \centering
            \includegraphics[width=\textwidth]{build/plots/energiekbTgleichzwei_sweep.pdf}
            \caption[]%
            {{\small Die Energie des Sytems nach der Äquilibrierungsphase bei zufällig geordneten Spins im Anfangszustand und $k_\text{B}T=2$.}}
            \label{fig:lcghist2}
        \end{subfigure}
        \vskip\baselineskip
        \begin{subfigure}[b]{0.475\textwidth}
            \centering
            \includegraphics[width=\textwidth]{build/plots/energiekbTgleichdrei_sweep.pdf}
            \caption[]%
            {{\small Die Energie des Sytems nach der Äquilibrierungsphase bei zufällig geordneten Spins im Anfangszustand und $k_\text{B}T=3$.}}
            \label{fig:lcghist3}
        \end{subfigure}


    \end{figure*}

    \begin{figure*}
            \centering
            \begin{subfigure}[b]{0.475\textwidth}
                \centering
                \includegraphics[width=\textwidth]{build/plots/magnetkbTgleicheins_sweep.pdf}
                \caption[Network2]%
                {{\small Die Magnetisierung des Sytems nach der Äquilibrierungsphase bei zufällig geordneten Spins im Anfangszustand und $k_\text{B}T=1$.}}
                \label{fig:lcghist1}
            \end{subfigure}
            \hfill
            \begin{subfigure}[b]{0.475\textwidth}
                \centering
                \includegraphics[width=\textwidth]{build/plots/magnetkbTgleichzwei_sweep.pdf}
                \caption[]%
                {{\small Die Magnetisierung des Sytems nach der Äquilibrierungsphase bei zufällig geordneten Spins im Anfangszustand und $k_\text{B}T=2$.}}
                \label{fig:lcghist2}
            \end{subfigure}
            \vskip\baselineskip
            \begin{subfigure}[b]{0.475\textwidth}
                \centering
                \includegraphics[width=\textwidth]{build/plots/magnetkbTgleichdrei_sweep.pdf}
                \caption[]%
                {{\small Die Magnetisierung des Sytems nach der Äquilibrierungsphase bei zufällig geordneten Spins im Anfangszustand und $k_\text{B}T=3$.}}
                \label{fig:lcghist3}
            \end{subfigure}


        \end{figure*}

        \begin{figure*}
                \centering
                \begin{subfigure}[b]{0.475\textwidth}
                    \centering
                    \includegraphics[width=\textwidth]{build/plots/absmagnetkbTgleicheins_sweep.pdf}
                    \caption[Network2]%
                    {{\small Der Betrag der Magnetisierung des Sytems nach der Äquilibrierungsphase bei zufällig geordneten Spins im Anfangszustand und $k_\text{B}T=1$.}}
                    \label{fig:lcghist1}
                \end{subfigure}
                \hfill
                \begin{subfigure}[b]{0.475\textwidth}
                    \centering
                    \includegraphics[width=\textwidth]{build/plots/absmagnetkbTgleichzwei_sweep.pdf}
                    \caption[]%
                    {{\small Der Betrag der Magnetisierung des Sytems nach der Äquilibrierungsphase bei zufällig geordneten Spins im Anfangszustand und $k_\text{B}T=2$.}}
                    \label{fig:lcghist2}
                \end{subfigure}
                \vskip\baselineskip
                \begin{subfigure}[b]{0.475\textwidth}
                    \centering
                    \includegraphics[width=\textwidth]{build/plots/absmagnetkbTgleichdrei_sweep.pdf}
                    \caption[]%
                    {{\small Der Betrag der Magnetisierung des Sytems nach der Äquilibrierungsphase bei zufällig geordneten Spins im Anfangszustand und $k_\text{B}T=3$.}}
                    \label{fig:lcghist3}
                \end{subfigure}


            \end{figure*}

            \begin{figure*}
                    \centering
                    \begin{subfigure}[b]{0.475\textwidth}
                        \centering
                        \includegraphics[width=\textwidth]{build/plots/tempmagnet.pdf}
                        \caption[Network2]%
                        {{\small Plot von $<m>(T)$ bei 1000 Sweeps.}}
                        \label{fig:lcghist1}
                    \end{subfigure}
                    \hfill
                    \begin{subfigure}[b]{0.475\textwidth}
                        \centering
                        \includegraphics[width=\textwidth]{build/plots/abstempmagnet.pdf}
                        \caption[]%
                        {{\small Plot von $<|m|>(T)$ bei 1000 Sweeps.}}
                        \label{fig:lcghist2}
                    \end{subfigure}
                \end{figure*}



\subsection*{d)}

\begin{figure}
  \includegraphics{build/plots/spezwaerme.pdf}
  \caption{Die spezifische Wärme pro Spin in Abhängigkeit von der Temperatur.}
  \label{}
\end{figure}








\end{document}
