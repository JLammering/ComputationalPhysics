\input{header.tex}

\usepackage{listings}
\usepackage[dvipsnames]{xcolor}

\title{Blatt 8}
\date{
  Abgabe: \today
}

\begin{document}
\maketitle
Wir stimmen für die mündliche Prüfung.

Alle Dateien ausführbar mit der Makefile. %Der Output wird dann in die Datei\texttt{blatt2\_ausgabe.txt} geschrieben.
\section*{Aufgabe 1}
\subsection*{a) , b)}
\begin{figure}
  \includegraphics[width=\textwidth]{cp_8_11.jpg}
  \caption{Aufgabe 1 a) und b) (1/2).}
  \label{fig:bild1}
\end{figure}


\begin{figure}
  \includegraphics[width=\textwidth]{cp_8_12.jpg}
  \caption{Aufgabe 1 a) und b) (2/2).}
  \label{fig:bild2}
\end{figure}

\FloatBarrier
\subsection*{c)}

Energieerhaltung nicht so ganz gegeben.
(Haben den (Tipp-)Fehler leider nicht gefunden.)

\begin{figure}
  \includegraphics[width=\textwidth]{build/energie.pdf}
  \caption{Aufgabe 1 c) 1. startbedingung.}
  \label{fig:bild3}
\end{figure}

\begin{figure}
  \includegraphics[width=\textwidth]{build/energie.pdf}
  \caption{Aufgabe 1 c) 2. startbedingung.}
  \label{fig:bild4}
\end{figure}


\begin{figure}
  \includegraphics[width=\textwidth]{build/1.pdf}
  \caption{Aufgabe 1 c) 1. startbedingung.}
  \label{fig:bild5}
\end{figure}

\begin{figure}
  \includegraphics[width=\textwidth]{build/2.pdf}
  \caption{Aufgabe 1 c) 2. startbedingung.}
  \label{fig:bild5}
\end{figure}


Animationen(.mp4) werden mit \textit{make} erstellt.



\end{document}
