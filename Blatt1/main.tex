\input{../header.tex}

\usepackage{listings}
\usepackage[dvipsnames]{xcolor}

\title{Blatt 1}
\date{
  Abgabe: 20.4.18
}

\begin{document}
\maketitle

Alle Dateien ausführbar mit der Makefile. Der Output wird dann in die Datei

\texttt{blatt1\_ausgabe.txt} geschrieben.
\section*{Aufgabe 1}

\begin{figure}
  \centering
  \includegraphics[height=9cm]{plot.pdf}
  \caption{Aufgabe 1a: $R_\text{N}(N)$}
  \label{fig:plotRN}
\end{figure}

Code des self avoiding randomwalk ist zu finden in \texttt{src/saw.h}. Dieser erzeugt die txt-Datei

 \texttt{mittlClustergroesse.txt}. Damit wird mit \texttt{plotRN.py} die Datei \texttt{plot.pdf} erzeugt.
In Abbildung \ref{fig:plotRN} ist der Plot zu sehen in dem die Wurzel aus der quadratischen mittleren Clustergröße $R_N$ gegen N aufgetragen ist.

Anschließend wird eine Kurve der Funktion
\begin{align}
  R_\text{N}(N) = a N^e
\end{align}
an die Werte gefittet. Die sich ergebenden Fitwerte sind ebenfalls in Abbildung \ref{fig:plotRN} zu finden.

\section*{Aufgabe 2}

\subsection*{a}
Der Code ist in der Datei \texttt{src/piAusKugel.h} zu finden.
Ergebnisse sind in der Ausgabe.

\subsection*{b}
Der Code ist in der Datei \texttt{src/uneigentlichesIntegral.h} zu finden.

Die Formel des Übungsblattes wird durch die Substitution $x = \tan(u)$ umgeformt zu:
\begin{align}
  \int_{-\frac{\pi}{2}}^{\arctan(T_i)} \frac{1}{\cos^2(u)} \frac{1}{\sqrt{\pi}} \exp{(-\tan^2(u))} \symup{d} u.
\end{align}
Ergebnisse sind in der Ausgabe.


\end{document}
