\input{header.tex}

\usepackage{listings}
\usepackage[dvipsnames]{xcolor}

\title{Blatt 7}
\date{
  Abgabe: 22.06.18
}

\begin{document}
\maketitle
Wir stimmen für die mündliche Prüfung.

Alle Dateien ausführbar mit der Makefile. %Der Output wird dann in die Datei\texttt{blatt2\_ausgabe.txt} geschrieben.
\section*{Aufgabe 1}
\subsection*{a)}
\begin{figure*}
        \centering
        \begin{subfigure}[b]{0.475\textwidth}
            \centering
            \includegraphics[width=\textwidth]{build/zeitauslenk102Schritten.pdf}
            \caption[Network2]%
            {{\small Die Auslenkung bei 102 Zeitschritten und rungekutta2.}}
            \label{fig:lcghist1}
        \end{subfigure}
        \hfill
        \begin{subfigure}[b]{0.475\textwidth}
            \centering
            \includegraphics[width=\textwidth]{build/zeitauslenk202Schritten.pdf}
            \caption[]%
            {{\small Die Auslenkung bei 202 Zeitschritten und rungekutta2.}}
            \label{fig:lcghist2}
        \end{subfigure}
        \vskip\baselineskip
        \begin{subfigure}[b]{0.475\textwidth}
            \centering
            \includegraphics[width=\textwidth]{build/zeitauslenk2002Schritten.pdf}
            \caption[]%
            {{\small Die Auslenkung bei 2002 Zeitschritten und rungekutta2.}}
            \label{fig:lcghist3}
        \end{subfigure}
        \quad
        \begin{subfigure}[b]{0.475\textwidth}
            \centering
            \includegraphics[width=\textwidth]{build/zeitauslenk5002Schritten.pdf}
            \caption[]%
            {{\small Die Auslenkung bei 5002 Zeitschritten und rungekutta2.}}
            \label{fig:lcghist4}
        \end{subfigure}
    \end{figure*}

    \begin{figure*}
            \centering
            \begin{subfigure}[b]{0.475\textwidth}
                \centering
                \includegraphics[width=\textwidth]{build/zeitauslenk103Schritten.pdf}
                \caption[Network2]%
                {{\small Die Auslenkung bei 103 Zeitschritten und euler.}}
                \label{fig:lcghist1}
            \end{subfigure}
            \hfill
            \begin{subfigure}[b]{0.475\textwidth}
                \centering
                \includegraphics[width=\textwidth]{build/zeitauslenk203Schritten.pdf}
                \caption[]%
                {{\small Die Auslenkung bei 203 Zeitschritten und euler.}}
                \label{fig:lcghist2}
            \end{subfigure}
            \vskip\baselineskip
            \begin{subfigure}[b]{0.475\textwidth}
                \centering
                \includegraphics[width=\textwidth]{build/zeitauslenk2003Schritten.pdf}
                \caption[]%
                {{\small Die Auslenkung bei 2003 Zeitschritten und euler.}}
                \label{fig:lcghist3}
            \end{subfigure}
            \quad
            \begin{subfigure}[b]{0.475\textwidth}
                \centering
                \includegraphics[width=\textwidth]{build/zeitauslenk5003Schritten.pdf}
                \caption[]%
                {{\small Die Auslenkung bei 5003 Zeitschritten und euler.}}
                \label{fig:lcghist4}
            \end{subfigure}
        \end{figure*}

\begin{figure*}
        \centering
        \begin{subfigure}[b]{0.475\textwidth}
            \centering
            \includegraphics[width=\textwidth]{build/zeitauslenk101Schritten.pdf}
            \caption[Network2]%
            {{\small Die Auslenkung bei 101 Zeitschritten und rungekutta4.}}
            \label{fig:lcghist1}
        \end{subfigure}
        \hfill
        \begin{subfigure}[b]{0.475\textwidth}
            \centering
            \includegraphics[width=\textwidth]{build/zeitauslenk201Schritten.pdf}
            \caption[]%
            {{\small Die Auslenkung bei 201 Zeitschritten und rungekutta4.}}
            \label{fig:lcghist2}
        \end{subfigure}
        \vskip\baselineskip
        \begin{subfigure}[b]{0.475\textwidth}
            \centering
            \includegraphics[width=\textwidth]{build/zeitauslenk2001Schritten.pdf}
            \caption[]%
            {{\small Die Auslenkung bei 2001 Zeitschritten und rungekutta4.}}
            \label{fig:lcghist3}
        \end{subfigure}
        \quad
        \begin{subfigure}[b]{0.475\textwidth}
            \centering
            \includegraphics[width=\textwidth]{build/zeitauslenk5001Schritten.pdf}
            \caption[]%
            {{\small Die Auslenkung bei 5001 Zeitschritten und rungekutta4.}}
            \label{fig:lcghist4}
        \end{subfigure}
    \end{figure*}

Die maximale Auslenkung wird bei rungekutta4 immer erreicht, aber bei kleinen Schrittweiten und Verfahren mit größeren Fehlerordnungen führen Rundungsfehler dazu, dass die Auslenkung sogar über eins hinausgeht.

\begin{figure*}
        \centering
        \begin{subfigure}[b]{0.475\textwidth}
            \centering
            \includegraphics[width=\textwidth]{build/energieerhaltung102Schritten.pdf}
            \caption[Network2]%
            {{\small Die Energieerhaltung bei 102 Zeitschritten und rungekutta2.}}
            \label{fig:lcghist1}
        \end{subfigure}
        \hfill
        \begin{subfigure}[b]{0.475\textwidth}
            \centering
            \includegraphics[width=\textwidth]{build/energieerhaltung202Schritten.pdf}
            \caption[]%
            {{\small Die Energieerhaltung bei 202 Zeitschritten und rungekutta2.}}
            \label{fig:lcghist2}
        \end{subfigure}
        \vskip\baselineskip
        \begin{subfigure}[b]{0.475\textwidth}
            \centering
            \includegraphics[width=\textwidth]{build/energieerhaltung4002Schritten.pdf}
            \caption[]%
            {{\small Die Energieerhaltung bei 4002 Zeitschritten und rungekutta2.}}
            \label{fig:lcghist3}
        \end{subfigure}
        \quad
        \begin{subfigure}[b]{0.475\textwidth}
            \centering
            \includegraphics[width=\textwidth]{build/energieerhaltung5002Schritten.pdf}
            \caption[]%
            {{\small Die Energieerhaltung bei 5002 Zeitschritten und rungekutta2.}}
            \label{fig:lcghist4}
        \end{subfigure}
    \end{figure*}

    \begin{figure*}
            \centering
            \begin{subfigure}[b]{0.475\textwidth}
                \centering
                \includegraphics[width=\textwidth]{build/energieerhaltung103Schritten.pdf}
                \caption[Network2]%
                {{\small Die Energieerhaltung bei 103 Zeitschritten und euler.}}
                \label{fig:lcghist1}
            \end{subfigure}
            \hfill
            \begin{subfigure}[b]{0.475\textwidth}
                \centering
                \includegraphics[width=\textwidth]{build/energieerhaltung203Schritten.pdf}
                \caption[]%
                {{\small Die Energieerhaltung bei 203 Zeitschritten und euler.}}
                \label{fig:lcghist2}
            \end{subfigure}
            \vskip\baselineskip
            \begin{subfigure}[b]{0.475\textwidth}
                \centering
                \includegraphics[width=\textwidth]{build/energieerhaltung4003Schritten.pdf}
                \caption[]%
                {{\small Die Energieerhaltung bei 4003 Zeitschritten und euler.}}
                \label{fig:lcghist3}
            \end{subfigure}
            \quad
            \begin{subfigure}[b]{0.475\textwidth}
                \centering
                \includegraphics[width=\textwidth]{build/energieerhaltung5003Schritten.pdf}
                \caption[]%
                {{\small Die Energieerhaltung bei 5003 Zeitschritten und euler.}}
                \label{fig:lcghist4}
            \end{subfigure}
        \end{figure*}


\begin{figure*}
        \centering
        \begin{subfigure}[b]{0.475\textwidth}
            \centering
            \includegraphics[width=\textwidth]{build/energieerhaltung101Schritten.pdf}
            \caption[Network2]%
            {{\small Die Energieerhaltung bei 101 Zeitschritten und rungekutta4.}}
            \label{fig:lcghist1}
        \end{subfigure}
        \hfill
        \begin{subfigure}[b]{0.475\textwidth}
            \centering
            \includegraphics[width=\textwidth]{build/energieerhaltung201Schritten.pdf}
            \caption[]%
            {{\small Die Energieerhaltung bei 201 Zeitschritten und rungekutta4.}}
            \label{fig:lcghist2}
        \end{subfigure}
        \vskip\baselineskip
        \begin{subfigure}[b]{0.475\textwidth}
            \centering
            \includegraphics[width=\textwidth]{build/energieerhaltung4001Schritten.pdf}
            \caption[]%
            {{\small Die Energieerhaltung bei 4001 Zeitschritten und rungekutta4.}}
            \label{fig:lcghist3}
        \end{subfigure}
        \quad
        \begin{subfigure}[b]{0.475\textwidth}
            \centering
            \includegraphics[width=\textwidth]{build/energieerhaltung5001Schritten.pdf}
            \caption[]%
            {{\small Die Energieerhaltung bei 5001 Zeitschritten und rungekutta4.}}
            \label{fig:lcghist4}
        \end{subfigure}
    \end{figure*}
Man erkennt, dass bei feinerer Schrittweite die Rundungsfehler kleiner werden und die Energieerhaltung besser erfüllt ist.
\subsection*{b)}

\begin{figure}
  \centering
  \includegraphics{build/4001Schritten.pdf}
  \caption{Zum Vergleich ein 3D-Plot mit $r=(1, 1, 1)$ und $v=(0, 0, 0)$.}
  \label{}
\end{figure}
\begin{figure}
  \centering
  \includegraphics{build/3000Schritten.pdf}
  \caption{Ein 3D-Plot mit $r=(1, 1, 1)$ und $v=(1, 1, 1)$.}
  \label{fig:4001Schritten}
\end{figure}
Abbildung \ref{fig:4001Schritten} zeigt, dass $v(0) \neq 0$ nur dazu führt, dass die Auslenkung größer wird, da dann $r(0)$ nicht mehr die maximale Auslenkung ist, sondern nur ein Punkt auf der Bahn.
\begin{figure}
  \centering
  \includegraphics{build/2999Schritten.pdf}
  \caption{Ein 3D-Plot mit $r=(1, 1, 1)$ und $v=(1, 0, 0)$.}
  \label{fig:2999Schritten}
\end{figure}
Abbildung \ref{fig:2999Schritten} zeigt, dass $r(0) \nparallel v(0)$ zu einer Kreisbahn führt.


\end{document}
