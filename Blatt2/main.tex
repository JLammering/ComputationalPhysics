\input{../header.tex}

\usepackage{listings}
\usepackage[dvipsnames]{xcolor}

\title{Blatt 1}
\date{
  Abgabe: 20.4.18
}

\begin{document}
\maketitle

Alle Dateien ausführbar mit der Makefile. %Der Output wird dann in die Datei\texttt{blatt2\_ausgabe.txt} geschrieben.
\section*{Aufgabe 1}

Code um Zufallszahlen zu erzeugen und in text-Dateien zu speichern ist in \texttt{aufgabe1/aufgabe1.cpp} zu finden. Alle Plots werden erzeugt mit \texttt{aufgabe1/plot.py}.

\subsection*{b}

\begin{figure*}
        \centering
        \begin{subfigure}[b]{0.475\textwidth}
            \centering
            \includegraphics[width=\textwidth]{build/ersterHist.pdf}
            \caption[Network2]%
            {{\small $r_0 = 1234,a = 20,c = 120,m = 6075$}}
            \label{fig:lcghist1}
        \end{subfigure}
        \hfill
        \begin{subfigure}[b]{0.475\textwidth}
            \centering
            \includegraphics[width=\textwidth]{build/zweiterHist.pdf}
            \caption[]%
            {{\small $r_0 = 1234,a = 137,c = 187,m = 256$}}
            \label{fig:lcghist2}
        \end{subfigure}
        \vskip\baselineskip
        \begin{subfigure}[b]{0.475\textwidth}
            \centering
            \includegraphics[width=\textwidth]{build/dritterHist.pdf}
            \caption[]%
            {{\small $r_0 = 123456789,a = 65539,c = 0,$\\$m = 2147483648$}}
            \label{fig:lcghist3}
        \end{subfigure}
        \quad
        \begin{subfigure}[b]{0.475\textwidth}
            \centering
            \includegraphics[width=\textwidth]{build/vierterHist.pdf}
            \caption[]%
            {{\small $r_0 = 1234,a = 16807,c = 0,$\\$m = 2147483647$}}
            \label{fig:lcghist4}
        \end{subfigure}
        \caption[ The average and standard deviation of critical parameters ]
        {\small Histogramme der linear kongruenten Generatoren mit den jeweiligen Parametern.}
        \label{fig:lcghist}
    \end{figure*}

Man erkennt in \ref{fig:lcghist1} und in \ref{fig:lcghist2} Periodizitäten in der Verteilung. Die Verteilungen in \ref{fig:lcghist3} und \ref{fig:lcghist4} sehen zufälliger aus, wobei \ref{fig:lcghist4} aber weniger gleichverteilt aussieht.

\subsection*{c}

\begin{figure*}
        \centering
        \begin{subfigure}[b]{0.475\textwidth}
            \centering
            \includegraphics[width=\textwidth]{build/ersterScatter.pdf}
            \caption[Network2]%
            {{\small $r_0 = 1234,a = 20,c = 120,m = 6075$}}
            \label{fig:lcgscatter1}
        \end{subfigure}
        \hfill
        \begin{subfigure}[b]{0.475\textwidth}
            \centering
            \includegraphics[width=\textwidth]{build/zweiterScatter.pdf}
            \caption[]%
            {{\small $r_0 = 1234,a = 137,c = 187,m = 256$}}
            \label{fig:lcgscatter2}
        \end{subfigure}
        \vskip\baselineskip
        \begin{subfigure}[b]{0.475\textwidth}
            \centering
            \includegraphics[width=\textwidth]{build/dritterScatter.pdf}
            \caption[]%
            {{\small $r_0 = 123456789,a = 65539,c = 0,$\\$m = 2147483648$}}
            \label{fig:lcgscatter3}
        \end{subfigure}
        \quad
        \begin{subfigure}[b]{0.475\textwidth}
            \centering
            \includegraphics[width=\textwidth]{build/vierterScatter.pdf}
            \caption[]%
            {{\small $r_0 = 1234,a = 16807,c = 0,$\\ $m = 2147483647$}}
            \label{fig:lcgscatter4}
        \end{subfigure}
        \caption[ The average and standard deviation of critical parameters ]
        {\small Korrelationsplots der linear kongruenten Generatoren mit den jeweiligen Parametern.}
        \label{fig:lcgscatter}
    \end{figure*}

Hier sieht man anhand der wenigen erreichten Datenpaare in \ref{fig:lcgscatter1} und \ref{fig:lcgscatter2} starke Korrelationen. \ref{fig:lcgscatter3} zeigt am wenigsten Korrelation.

\subsection*{d}

\begin{figure*}
        \centering
        \begin{subfigure}[b]{0.475\textwidth}
            \centering
            \includegraphics[width=\textwidth]{build/fifthHist.pdf}
            \caption[Network2]%
            {{\small $(a_1, b_1, c_1) = (11, 1, 7)$}}
            \label{fig:xorhist5}
        \end{subfigure}
        \hfill
        \begin{subfigure}[b]{0.475\textwidth}
            \centering
            \includegraphics[width=\textwidth]{build/fifthScatter.pdf}
            \caption[]%
            {{\small $(a_1, b_1, c_1) = (11, 1, 7)$}}
            \label{fig:xorscatter5}
        \end{subfigure}
        \vskip\baselineskip
        \begin{subfigure}[b]{0.475\textwidth}
            \centering
            \includegraphics[width=\textwidth]{build/sixHist.pdf}
            \caption[]%
            {{\small $(a_1, b_1, c_1) = (11, 4, 7)$}}
            \label{fig:xorhist6}
        \end{subfigure}
        \quad
        \begin{subfigure}[b]{0.475\textwidth}
            \centering
            \includegraphics[width=\textwidth]{build/sixScatter.pdf}
            \caption[]%
            {{\small $(a_1, b_1, c_1) = (11, 4, 7)$}}
            \label{fig:xorscatter6}
        \end{subfigure}
        \caption[ The average and standard deviation of critical parameters ]
        {\small Histogramme und Korrelationsplots zu den beiden XOR Shift Algorithmen mit ihren jeweiligen Parametern.}
        \label{fig:xor}
    \end{figure*}

    Erkennbar ist, dass die Parameter-Kombination $(a_1, b_1, c_1) = (11, 1, 7)$ ein deutlich gleichmäßigeres Histogramm liefert und mehr Datenpaare erreicht werden, als bei der Parameter-Kombination $(a_1, b_1, c_1) = (11, 4, 7)$.

\subsection*{e}

\begin{figure}
  \centering
  \includegraphics[height=5cm]{build/rekursionstest.pdf}
  \caption{Plot bei dem die Farbe der Vierecke die Periodenlänge der XORshift-Algorithmen mit der jeweiligen Parameterkombination angibt. Mit $a=11$ und $r_0=123$.}
  \label{fig:rekursionstest}
\end{figure}

Erkennbar ist, dass die verschiedenen Parameterkombination sehr verschiedene Periodenlängen ergeben. Die Periodenlänge wird oft als ein Qualitätskriterium eines Zufallszahlengenerators betrachtet.


% \begin{figure}
%   \centering
%   \includegraphics[height=9cm]{plot.pdf}
%   \caption{Aufgabe 1a: $R_\text{N}(N)$}
%   \label{fig:plotRN}
% \end{figure}


\section*{Aufgabe 2}

\subsection*{a}


\subsection*{b}


\end{document}
