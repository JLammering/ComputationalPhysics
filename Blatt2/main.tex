\input{../header.tex}

\usepackage{listings}
\usepackage[dvipsnames]{xcolor}

\title{Blatt 1}
\date{
  Abgabe: 04.05.18
}

\begin{document}
\maketitle

Alle Dateien ausführbar mit der Makefile. %Der Output wird dann in die Datei\texttt{blatt2\_ausgabe.txt} geschrieben.
\section*{Aufgabe 1}

Code um Zufallszahlen zu erzeugen und in text-Dateien zu speichern ist in \texttt{aufgabe1/aufgabe1.cpp} zu finden. Alle Plots werden erzeugt mit \texttt{aufgabe1/plot.py}.

\subsection*{b}

\begin{figure*}
        \centering
        \begin{subfigure}[b]{0.475\textwidth}
            \centering
            \includegraphics[width=\textwidth]{build/ersterHist.pdf}
            \caption[Network2]%
            {{\small $r_0 = 1234,a = 20,c = 120,m = 6075$}}
            \label{fig:lcghist1}
        \end{subfigure}
        \hfill
        \begin{subfigure}[b]{0.475\textwidth}
            \centering
            \includegraphics[width=\textwidth]{build/zweiterHist.pdf}
            \caption[]%
            {{\small $r_0 = 1234,a = 137,c = 187,m = 256$}}
            \label{fig:lcghist2}
        \end{subfigure}
        \vskip\baselineskip
        \begin{subfigure}[b]{0.475\textwidth}
            \centering
            \includegraphics[width=\textwidth]{build/dritterHist.pdf}
            \caption[]%
            {{\small $r_0 = 123456789,a = 65539,c = 0,$\\$m = 2147483648$}}
            \label{fig:lcghist3}
        \end{subfigure}
        \quad
        \begin{subfigure}[b]{0.475\textwidth}
            \centering
            \includegraphics[width=\textwidth]{build/vierterHist.pdf}
            \caption[]%
            {{\small $r_0 = 1234,a = 16807,c = 0,$\\$m = 2147483647$}}
            \label{fig:lcghist4}
        \end{subfigure}
        \caption[ The average and standard deviation of critical parameters ]
        {\small Histogramme der linear kongruenten Generatoren mit den jeweiligen Parametern.}
        \label{fig:lcghist}
    \end{figure*}

Man erkennt in \ref{fig:lcghist1} und in \ref{fig:lcghist2} Periodizitäten in der Verteilung. Die Verteilungen in \ref{fig:lcghist3} und \ref{fig:lcghist4} sehen zufälliger aus, wobei \ref{fig:lcghist4} aber weniger gleichverteilt aussieht.

\subsection*{c}

\begin{figure*}
        \centering
        \begin{subfigure}[b]{0.475\textwidth}
            \centering
            \includegraphics[width=\textwidth]{build/ersterScatter.pdf}
            \caption[Network2]%
            {{\small $r_0 = 1234,a = 20,c = 120,m = 6075$}}
            \label{fig:lcgscatter1}
        \end{subfigure}
        \hfill
        \begin{subfigure}[b]{0.475\textwidth}
            \centering
            \includegraphics[width=\textwidth]{build/zweiterScatter.pdf}
            \caption[]%
            {{\small $r_0 = 1234,a = 137,c = 187,m = 256$}}
            \label{fig:lcgscatter2}
        \end{subfigure}
        \vskip\baselineskip
        \begin{subfigure}[b]{0.475\textwidth}
            \centering
            \includegraphics[width=\textwidth]{build/dritterScatter.pdf}
            \caption[]%
            {{\small $r_0 = 123456789,a = 65539,c = 0,$\\$m = 2147483648$}}
            \label{fig:lcgscatter3}
        \end{subfigure}
        \quad
        \begin{subfigure}[b]{0.475\textwidth}
            \centering
            \includegraphics[width=\textwidth]{build/vierterScatter.pdf}
            \caption[]%
            {{\small $r_0 = 1234,a = 16807,c = 0,$\\ $m = 2147483647$}}
            \label{fig:lcgscatter4}
        \end{subfigure}
        \caption[ The average and standard deviation of critical parameters ]
        {\small Korrelationsplots der linear kongruenten Generatoren mit den jeweiligen Parametern.}
        \label{fig:lcgscatter}
    \end{figure*}

Hier sieht man anhand der wenigen erreichten Datenpaare in \ref{fig:lcgscatter1} und \ref{fig:lcgscatter2} starke Korrelationen. \ref{fig:lcgscatter3} zeigt am wenigsten Korrelation.

\subsection*{d}

\begin{figure*}
        \centering
        \begin{subfigure}[b]{0.475\textwidth}
            \centering
            \includegraphics[width=\textwidth]{build/fifthHist.pdf}
            \caption[Network2]%
            {{\small $(a_1, b_1, c_1) = (11, 1, 7)$}}
            \label{fig:xorhist5}
        \end{subfigure}
        \hfill
        \begin{subfigure}[b]{0.475\textwidth}
            \centering
            \includegraphics[width=\textwidth]{build/fifthScatter.pdf}
            \caption[]%
            {{\small $(a_1, b_1, c_1) = (11, 1, 7)$}}
            \label{fig:xorscatter5}
        \end{subfigure}
        \vskip\baselineskip
        \begin{subfigure}[b]{0.475\textwidth}
            \centering
            \includegraphics[width=\textwidth]{build/sixHist.pdf}
            \caption[]%
            {{\small $(a_1, b_1, c_1) = (11, 4, 7)$}}
            \label{fig:xorhist6}
        \end{subfigure}
        \quad
        \begin{subfigure}[b]{0.475\textwidth}
            \centering
            \includegraphics[width=\textwidth]{build/sixScatter.pdf}
            \caption[]%
            {{\small $(a_1, b_1, c_1) = (11, 4, 7)$}}
            \label{fig:xorscatter6}
        \end{subfigure}
        \caption[ The average and standard deviation of critical parameters ]
        {\small Histogramme und Korrelationsplots zu den beiden XOR Shift Algorithmen mit ihren jeweiligen Parametern.}
        \label{fig:xor}
    \end{figure*}

    Erkennbar ist, dass die Parameter-Kombination $(a_1, b_1, c_1) = (11, 1, 7)$ ein deutlich gleichmäßigeres Histogramm liefert und mehr Datenpaare erreicht werden, als bei der Parameter-Kombination $(a_1, b_1, c_1) = (11, 4, 7)$.

\subsection*{e}

\begin{figure}
  \centering
  \includegraphics[height=5cm]{build/rekursionstest.pdf}
  \caption{Plot bei dem die Farbe der Vierecke die Periodenlänge der XORshift-Algorithmen mit der jeweiligen Parameterkombination angibt. Mit $a=11$ und $r_0=123$.}
  \label{fig:rekursionstest}
\end{figure}

Erkennbar ist, dass die verschiedenen Parameterkombination sehr verschiedene Periodenlängen ergeben. Die Periodenlänge wird oft als ein Qualitätskriterium eines Zufallszahlengenerators betrachtet.


% \begin{figure}
%   \centering
%   \includegraphics[height=9cm]{plot.pdf}
%   \caption{Aufgabe 1a: $R_\text{N}(N)$}
%   \label{fig:plotRN}
% \end{figure}


\section*{Aufgabe 2}

\subsection*{a}
\subsubsection*{(i)}
Um aus gleichverteilten Zufallszahlen(u(x) = 1) auf dem Intervall $[0,1]$, Zufallszahlen
der Dichtefunktion $p(y) = cos(y)$ zu erzeugen, wird die Transformationsmethode
wie folgt angewendet:
\begin{align*}
  \frac{1}{\left(U(1) - U(0)\right)} \int_{0}^{x} u(x^\prime) \mathrm{d}^\prime &= \frac{1}{\left( \sin \pi /2) - \sin 0 \right) }\int_{0}^{y} cos(y^\prime) \mathrm{d}y^\prime\\
  x &= \sin y \\
  y &= \arcsin x
\end{align*}
Dabei wird das Zielintervall so gewählt, dass die Sinusfunktion keine negativen Werte annimmt. Es wird also das Intervall
zwischen $0$ und $\pi / 2$ verwendet.\\ \\
In Abbildung \ref{fig:a2_a_i} sind die generierten Zufallszahlen histogrammiert zusammen mit der Kosinusfunktion auf dem Intervall zwischen $0$ $\pi / 2$ dargestellt.

\subsubsection*{(ii)}
Um aus gleichverteilten Zufallszahlen(u(x) = 1) auf dem Intervall $[-1,1]$, Zufallszahlen
der Dichtefunktion $p(y) = cos(y)$ zu erzeugen, wird die Transformationsmethode
wie folgt angewendet:
\begin{align*}
  \frac{1}{\left(U(1) - U(-1)\right)} \int_{-1}^{x} u(x^\prime) \mathrm{d}^\prime &= \frac{1}{\left( \sin \pi /2) - \sin 0 \right) }\int_{0}^{y} cos(y^\prime) \mathrm{d}y^\prime\\
  \frac{1}{2} \left( x + 1 \right) &= \sin y \\
  y &= \arcsin \frac{1}{2} \left( x + 1 \right)
\end{align*}
Dabei wird wieder das Zielintervall so gewählt, dass die Sinusfunktion keine negativen Werte annimmt. Es wird also das Intervall
zwischen $0$ und $\pi / 2$ verwendet.
In Abbildung \ref{fig:a2_a_ii} sind die generierten Zufallszahlen histogrammiert zusammen mit der Kosinusfunktion auf dem Intervall zwischen $0$ $\pi / 2$ dargestellt. \\ \\
Beim vergleichen der Abbildungen \ref{fig:a2_a_i} und \ref{fig:a2_ii}, wird klar, dass die selbe Verteilung generiert wird.
Allerdings

\begin{figure}[h]
  \centering
  \includegraphics[width=\textwidth]{build/aufgabe2_a_i.pdf}
  \caption{Kosinusverteilte Zufallszahlen erzeugt aus gleichverteilten Zufallszahlen zwischen $-1$ und $1$.}
  \label{fig:a2_a_i}
\end{figure}

\begin{figure}[h]
  \centering
  \includegraphics[width=\textwidth]{build/aufgabe2_a_ii.pdf}
  \caption{Kosinusverteilte Zufallszahlen erzeugt aus gleichverteilten Zufallszahlen zwischen $-1$ und $1$.}
  \label{fig:a2_a_ii}
\end{figure}


\subsection*{b}
Mit der Box-Muller-Methode werden mithilfe von zwei gleichverteilten reelen Zufallszahlen $u_1$, $u_2$, aus dem Intervall zwischen
$0$ und $1$, zwei Standardnormalverteilte Zufallszahlen $x$, $y$ erzeugt. \\
Dies erfolgt gemäß
\begin{align*}
  x &= \cos 2\pi u_1 \sqrt{-2 \ln u_2} \\
  y &= \sin 2\pi u_1 \sqrt{-2 \ln u_2} .
\end{align*}

Da es sich um die so erzeugten Zufallszahlen aber nur um eine Verteilung mit Mittelwert $\mu = 0$ und Standardabweichung $\sigma = 1$ handelt,
müssen die einzelnen Zahlen $x_i$ (oder auch $y_i$) noch in Zahlen $z_i$ der gewünschten Verteilung mit $ \mu = 3$ und $\sigma^2 = 4$ gemäß
\begin{align*}
  z_i &= \sigma x_i + \mu \\
  z_i &= 4 x_i + 3
\end{align*}
überführt werden.

In Abbildung \ref{fig:a2_b} sind die mit der Box-Muller-Methode erzeugten Zufallszahlen histogrammiert zusammen mit der passenden Normalverteilung
dargestellt. Dabei beträgt die Zahl der generierten Zufallszahlen $10^{6}$, die Zahl der Bins $\sqrt{10^6} = 10^3$ und das gewählte Intervall $[-40, 40]$.

\begin{figure}[h]
  \centering
  \includegraphics[width=\textwidth]{build/aufgabe2_b.pdf}
  \caption{Normalverteilte Zufallszahlen ($\mu = 3$, $\sigma = 4$.}
  \label{fig:a2_b}
\end{figure}


\end{document}
